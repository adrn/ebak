\documentclass[12pt,letterpaper,preprint]{aastex}

% Bringhurst
\linespread{1.08}
\setlength{\parindent}{1.08\baselineskip}
\frenchspacing

\newcommand{\foreign}[1]{\textsl{#1}}
\newcommand{\etal}{\foreign{et~al.}}
\newcommand{\acronym}[1]{{\small{#1}}}
\newcommand{\project}[1]{\textsl{#1}}
\newcommand{\kepler}{\project{Kepler}}
\newcommand{\apogee}{\project{\acronym{APOGEE}}}

\begin{document}

\title{Eclipsing binary star characterization with \kepler\ and \apogee}
\author{
  Adrian~Price-Whelan~(Columbia,~Princeton),
  David~W.~Hogg~(SCDA,~NYU,~MPIA),
  Melissa~Ness~(MPIA),
  Daniel~Foreman-Mackey~(UW), others}

\begin{abstract}
Long-period eclipsing binaries potentially provide extremely precise
tests of stellar models; with spectroscopic coverage, it is possible
to determine absolute radii and masses for both stars; at longer
periods it is less likely that accretion or interaction has affected
the stars; in the likely event that the stars formed together, they
ought to have identical chemical abundances.
There are XXX stars in the overlap between the \kepler\ eclipsing
binary sample and the \apogee\ \acronym{DR12} spectroscopy.
Here we choose one of these systems---KIC~YYY or 2MZZZ---and construct
full posterior information about the stellar and orbital properties of
the system.
We find AAA and BBB.
This demonstrates that the combination of \kepler\ and \apogee\ data
is capable of delivering absolute physical properties for the system.
The prospects for automatically characterizing all such systems---and
using the output to build new kinds of data-driven models for
stars---are discussed.
\end{abstract}

Hello World?

\begin{thebibliography}{24}\raggedright
\bibitem[Gaulme \etal(2013)]{gaulme}
  Gaulme, P., McKeever, J., Rawls, M.~L., \etal, 2013, \apj, 767, 82 
\bibitem[Rawls \etal(2016)]{rawls}
  Rawls, M.~L., Gaulme, P., McKeever, J., \etal, 2016, \apj, 818, 108 
\bibitem[Shapley(1913)]{shapley}
  Shapley, H., 1913, \apj, 38, 158
\bibitem[Troup \etal(2016)]{troup}
  Troup, N.~W., Nidever, D.~L., De Lee, N., \etal, 2016, \aj, 151, 85 
\bibitem[Winn(2010)]{winn}
  Winn, J.~N., 2010, arXiv:1001.2010
\end{thebibliography}

\end{document}
