\documentclass[12pt, preprint]{aastex}

\newcommand{\project}[1]{\textsl{#1}}
\newcommand{\acronym}[1]{{\small{#1}}}
\newcommand{\apogee}{\project{\acronym{APOGEE}}}
\newcommand{\dr}{\acronym{DR13}}

\newcommand{\meterspersecond}{\mathrm{m\,s^{-1}}}

\begin{document}

\title{A custom Monte Carlo sampler
  for binary-star and exoplanet radial velocity data}
\author{APW, DWH, DFM, others}

\begin{abstract}
% Context
Given sparse radial-velocity measurements of a star, there are often
many qualitatively different stellar or exoplanet companion orbit
models that are consistent with the data.
The consequent multimodality of the likelihood function leads to
extremely challenging search, optimization, and MCMC posterior
sampling in the space of orbital parameters.
% Aims
Here we create a custom-build Monte Carlo sampler that can produce a
posterior sampling for orbital parameters given even small numbers of
noisy radial-valocity measurements (and hence very complex likelihood
function).
The goal is to obtain provably correct samplings in the space of
orbital parameters.
% Methods
We capitalize on the similarity between orbit fitting and linear
sinusoid fitting, where we can---at any given orbital
period---generate a perfect sampling because of the problem structure.
We transform the perfect samplings in the linear problem into a
perfect sampling in the real orbital problem using importance
sampling.
% Results
We find that we can quickly produce correct samplings in orbital
parameters for data sets that include as few as three noisy time
points.
Although the method produces correct samplings by construction, the
method becomes inefficient (at the importance sampling stage) when the
eccentricity becomes large, because in this case the linear and real
problems diverge substantially.
\end{abstract}

\keywords{
  ---
  Hello
  ---
  World: Hello
  ---
}

Hello World.

\end{document}
