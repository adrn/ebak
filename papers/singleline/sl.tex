\documentclass[12pt, preprint]{aastex}

\newcommand{\project}[1]{\textsl{#1}}
\newcommand{\acronym}[1]{{\small{#1}}}
\newcommand{\apogee}{\project{\acronym{APOGEE}}}
\newcommand{\dr}{\acronym{DR13}}

\newcommand{\meterspersecond}{\mathrm{m\,s^{-1}}}

\begin{document}

\title{Binary orbital parameters and population statistics
  from stars with only a few radial velocity measurements each}
\author{APW, DWH, HWR, others}

\begin{abstract}
With sufficient time coverage and velocity precision, it is possible
to determine (up to an inclination) the orbital parameters of a binary
star system, with measurements only of the radial velocity of one of
the two stars.
In the context of new spectroscopy projects with limited time-domain
coverage, it is important to find ways to make these determinations
with fewer and more imperfect observations, and nonetheless learn
important things about the binary stars and their population.
Here we create a likelihood function, including a stellar noise model and
an observation outlier (catastrophic error) model, and demonstrate
that it is possible to perform probabilistic inference with Monte
Carlo methods under vague (interim) priors.
We also use an importance-sampling technique we have used previously
in exoplanet contexts to perform a hierarchical inference of the
properties of the full population of binary star systems, given only
(sometimes fairly uninformative) posterior samplings under the interim
priors.
We deploy this inference at scale on data on 83,661 stars taken from
\apogee\ \dr\, with a minimum of YY and an average of XX
radial-velocity measurements per star, with typical uncertainties of
about $300\,\meterspersecond$.
We perform only a limited hierarchical inference and find ZZ and WW.
We compare our results to literature results for the same stars and
find that especially with small numbers of epochs ($<8$)---but even
with large numbers of epochs---there are often multiple qualitatively
different but nonetheless reasonable orbital explanations for any
individual system.
\end{abstract}

\keywords{
  ---
  Hello
  ---
  World: Hello
  ---
}

\section{Introduction} \label{sec:intro}

Hello World.

\section{\apogee data} \label{sec:data}

\section{Probabilistic model}\label{sec:prob-model}

\subsection{Likelihood function} \label{sec:likelihood}

\subsection{Priors} \label{sec:priors}

\end{document}
