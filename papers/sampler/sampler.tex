\documentclass[12pt, preprint]{aastex}

% to-do list
% ----------
% - zeroth draft

% style notes
% -----------

% fix aastex
\setlength{\parindent}{1.4em}
\linespread{1.15} % close to 10/13 spacing in ``manuscript''
\setlength{\parskip}{0ex}
\makeatletter % you know you are living your life wrong when you need to do this
\def\revtex@pageid{%
 \xdef\@thefnmark{\null}%
 \@footnotetext{%
  This \revtex@genre\space was prepared with the
  \revtex@org\space \LaTeX\ macros v\revtex@ver, with modifications by David~W.~Hogg.%
 }%
}%
\def\abstractname{Abstract}%
\makeatother
\sloppy\sloppypar\frenchspacing

% define macros for text
\newcommand{\project}[1]{\textsl{#1}}
\newcommand{\acronym}[1]{{\small{#1}}}
\newcommand{\apogee}{\project{\acronym{APOGEE}}}
\newcommand{\dr}{\acronym{DR13}}

% define macros for math
\newcommand{\meterspersecond}{\mathrm{m\,s^{-1}}}
\newcommand{\asini}{a\,\sin i}

\begin{document}

\title{A custom Monte Carlo sampler
  for binary-star and exoplanet radial velocity data}
\author{Adrian~M.~Price-Whelan\altaffilmark{\pu,\adrn},
        David~W.~Hogg\altaffilmark{\ccpp,\mpia},
        Daniel~Foreman-Mackey\altaffilmark{\uw,\sagan},
        Hans-Walter~Rix\altaffilmark{\mpia}
}

% Affiliations
\newcommand{\pu}{1}
\newcommand{\adrn}{2}
\newcommand{\ccpp}{3}
\newcommand{\mpia}{4}
\newcommand{\uw}{5}
\newcommand{\sagan}{6}

\altaffiltext{\pu}{Department of Astrophysical Sciences,
                   Princeton University, Princeton, NJ 08544, USA}
\altaffiltext{\adrn}{To whom correspondence should be addressed:
                     \texttt{adrn@princeton.edu}}
\altaffiltext{\ccpp}{Center for Cosmology and Particle Physics,
                     Department of Physics,
                     New York University, 4 Washington Place,
                     New York, NY 10003, USA}
\altaffiltext{\mpia}{Max-Planck-Institut f\"ur Astronomie,
                     K\"onigstuhl 17, D-69117 Heidelberg, Germany}
\altaffiltext{\uw}{Astronomy Department, University of Washington,
                   Seattle, WA 98195, USA}
\altaffiltext{\sagan}{Sagan Fellow}

\begin{abstract}
% Context
Given sparse radial-velocity measurements of a star, there are often
many qualitatively different stellar or exoplanet companion orbit
models that are consistent with the data.
The consequent multimodality of the likelihood function leads to
extremely challenging search, optimization, and MCMC posterior
sampling in the space of orbital parameters.
% Aims
Here we create a custom-build Monte Carlo sampler that can produce a
posterior sampling for orbital parameters given even small numbers of
noisy radial-valocity measurements (and hence very complex likelihood
function).
The goal is to obtain provably correct samplings in the space of
orbital parameters.
% Methods
We build a variant of simple Monte Carlo sampling, in which we densely
sample the non-linear orbital parameters, and perform rejection
sampling using a marginalized likelihood, marginalizing out the linear
orbital parameters.
In practice and generically, the sampling obtained by the simple Monte
Carlo is---in the case of sparse or uninformative data---substantial
and multimodal, or else---in the case of informative
data---insubstantial but unimodal.
In the unimodal case, we follow the simple Monte Carlo with standard
Markov Chain Monte Carlo to make the sampling more substantial.
% Results
The method produces correct samplings in orbital
parameters for (good) data sets that include as few as three noisy time
points.
We give some examples that show how the posterior probability depends
extremely strongly on the number and time coverage of the
observations.
We discuss extensions to the method that could handle issues with the
noise model or outliers.
\end{abstract}

\keywords{
  binaries: spectroscopic
  ---
  methods: data analysis
  ---
  methods: statistical
  ---
  planets and satellites: fundamental parameters
  ---
  surveys
  ---
  techniques: radial velocities
}

\section{Introduction}

Precise radial-velocity measurements of stars have transformed
astrophysics in the last decades:
They have permitted the discovery of the first planets around other stars,
including especially the unanticipated but common hot jupiters,
and been used to discover or confirm hundreds
(perhaps thousands?) of planets.
Radial velocity measurements have also been used to find substellar,
degenerate, and black-hole companions to more normal stars, and hold
the promise of delivering the full population statistics for binary
(and trinary) star systems.

With many new stellar spectroscopic surveys operating or under
construction, we expect to have good quality spectra for millions
of stars in the next few years.
Most of these surveys have at least some targets---and many have many
targets---that get observed multiple times.
These surveys can (as an auxilliary or primary goal of their observing
strategies) generate discoveries of planetary, substellar, and stellar
companions.
These discoveries, in turn, will feed population inferences, follow-up
programs, and projects to refine precise stellar models.

However, when radial-velocity observations are not designed with
unambiguous detection and discovery in mind, usually there are
multiple possible binary-star models that are consistent with any
small number of radial-velocity measurements that show stellar
acceleration.
That is, a small number of (even very good) radial velocity
measurements will lead to posterior beliefs about companion orbits and
masses that put substantial plausibility onto multiple qualitatively
different solutions, or (in other words) create a likelihood function
that is highly multi-model in the relevant parameter spaces.
There are currently no safe methods known for exploring these highly
multimodal functions and delivering correct posterior samplings and
reliable probabilistic statements about detection and
characterization.

Here we make an attempt at correcting these problems.
Our approach is to build custom posterior sampling methods that
capitalize on the structure of the binary-star (or star--exoplanet)
kinematics to create provably---or highly probably---correct
samplings.

The structure of the paper is as follows:
We state clearly our assumptions, and demonstrate that we have a
method that is correct under those assumptions.
We perform experiments with the method to understand its properties
and limitations.
We finish by discussing the value of the method, and the changes we
would have to make if we weakened our assumptions, or if we don't
weaken our assumptions but they indeed prove to be far from correct.

\section{Assumptions and method}

In order to set up a well-posed problem and build a path to a
definite solution, we make a set of non-trivial assumptions about the
stellar systems we are observing.
\begin{enumerate}
\item We assume that we have measurements of the radial velocity of a
  star, and that the time dependence of the expectation of that radial
  velocity is well described by the gravitational orbit of a pair of
  point masses (the Kepler problem). We assume that the times are
  (effectively) perfectly known, and in an inertial frame (for
  example, Solar-System barycentric MJD).
\item We assume that each star has only zero or one companions, and
  that the radial-velocity measurements are not contaminated by nor
  affected by any other bodies.
\item We assume that the noise contributions to individual
  radial-velocity measurements are well described as draws from
  zero-mean normal (Gaussian) distributions with correctly known
  variances. We assume that there are no outliers.
\item In addition to all these, we put particular, fairly sensible
  prior probability density functions on all the orbital parameters,
  described below.
\end{enumerate}
Each of these assumptions can be challenged, and in particular we
expect some stars to have additional companions, and we expect there
to be outliers and unaccounted sources of noise.
We will return to these assumptions, and the consequences of relaxing
them, in the Discussion Section.

In the Kepler problem (and here we are working similar to
\citealt{winn}), and under our above-stated assumptions, the
radial-velocity expectation can be parameterized by six parameters,
which we choose to be period $P$, projected semi-major axis $\asini$,
a phase $\phi_0$ corresponding to a time of pericenter passage, the
eccentricity $e$, an argument of perihelion $\varpi$, and a constant 
system barycenter radial velocity $v_0$.
Of these parameters, four ($P$, $\phi_0$, $e$, $\varpi$) are
non-linearly related to the radial-velocity expectation, and two
($\asini$, $v_0$) are linearly related.
Fundamentally, our method is to perform \emph{simple Monte Carlo} in
the non-linear parameters, and analytically marginalize out the linear
parameters.

What is simple monte carlo?

What are our explicit priors?

What do we do after SMC?

\section{Experiments and results}

\section{Discussion}

\acknowledgements
It is a pleasure to thank
  Ben Weaver (NOAO),
for valuable discussions.
This research was partially supported by [many grants].
This project was started at AstroHackWeek 2016, organized by Kyle
Barbary (UCB) and Phil Marshall (SLAC) at the Berkeley Institute for
Data Science.

\begin{thebibliography}{99}\raggedright
\bibitem[Kipping(2013)]{kipping}
  Kipping, D.~M.\ 2013, \mnras, 434, L51
\bibitem[Winn(2010)]{winn}
  Winn, J.~N.\ 2010, Exoplanets, 55 
\end{thebibliography}

\end{document}
